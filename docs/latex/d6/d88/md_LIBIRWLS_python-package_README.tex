To use this module you must have the libraries numpy and Cython installed in python.


\begin{DoxyItemize}
\item \href{http://cython.org/}{\tt Cython}
\item \href{http://www.numpy.org/}{\tt Numpy}
\end{DoxyItemize}

You can install very easily both libraries using the command \href{https://pip.pypa.io/en/stable/}{\tt pip}.

\subsection*{Installation Instructions\+:}

\subsubsection*{Requeriments\+:}

You need to build the application before installing the python extension. To do that follow the instructions in the main R\+E\+A\+D\+M\+E.\+md file.

\subsubsection*{Windows}

Currently this python extension is still not available for Windows operating systems.

\subsubsection*{Installation Linux, Unix}

If you have manually installed A\+T\+L\+AS (instead of using the apt-\/get command), you must tell the installation directory by defining the environment variable A\+T\+L\+A\+S\+D\+IR\+: \begin{DoxyVerb}export ATLASDIR=/path/to/atlas/
\end{DoxyVerb}


After that, you can easily install this python module (If you need sudo permission you have to use the parameter -\/E to keep the environment variable)\+: \begin{DoxyVerb}sudo -E python setup.py install
\end{DoxyVerb}


\subsubsection*{OS X}

You must define some enviroment variables\+:

CC to tell the compiler (use the same compiler that you use to compile the command line app). For example, in the case of gcc 6 installed with homebrew\+: \begin{DoxyVerb}export CC=/usr/local/Cellar/gcc/6.2.0/bin/gcc-6
\end{DoxyVerb}


Only if veclib is not in the default directory you must define the environment variable V\+E\+C\+L\+I\+B\+D\+IR telling where veclib is\+: \begin{DoxyVerb}export VECLIBDIR=/path/to/veclib
\end{DoxyVerb}


Your compiler has the openmp functions in the libgomp library, you must define an enviroment variable called L\+I\+B\+G\+O\+M\+P\+\_\+\+P\+A\+TH telling where is the file libgomp.\+a in the lib directory of your compiler. In the case of gcc 6 installed with homebrew\+: \begin{DoxyVerb}export LIBGOMP_PATH=/usr/local/Cellar/gcc/6.2.0/lib/gcc/6/libgomp.a
\end{DoxyVerb}


To avoid that Cython could use any posible flag only available for clang it is better to set the C\+F\+L\+A\+GS environment variable to an empty value\+: \begin{DoxyVerb}export CFLAGS=
\end{DoxyVerb}


With these environment variables well defined you can install the extension (If you need sudo permission you have to use the parameter -\/E to keep the environment variables)\+: \begin{DoxyVerb}sudo -E python setupOSX.py install
\end{DoxyVerb}


\subsection*{R\+U\+N\+N\+I\+NG\+:}

\subsubsection*{Import this library\+:}

\begin{DoxyVerb}    import LIBIRWLS
\end{DoxyVerb}


\subsubsection*{P\+I\+R\+W\+LS algorithm\+:}

It trains a S\+VM using a parallel I\+R\+W\+LS procedure. See the library \href{https://robedm.github.io/LIBIRWLS/}{\tt webpage} for a detailed description.

\begin{DoxyVerb}model = LIBIRWLS.PIRWLStrain(data, labels, gamma=1, C=1, threads=1, workingSet=500, eta=0.001, kernel=1)
\end{DoxyVerb}


Parameters\+:
\begin{DoxyItemize}
\item data\+: Training set (numpy 2d array)
\item labels\+: Training set labels (the label of every training data numpy array with values +1 and -\/1)
\item kernel\+: kernel type\+:
\begin{DoxyItemize}
\item 0 for Linear kernel u\textquotesingle{}{\itshape v}
\item {\itshape 1 for radial basis function exp(-\/gamma}$\vert$u-\/v$\vert$$^\wedge$2)
\end{DoxyItemize}
\item gamma\+: gamma in the radial basis kernel function
\item C\+: S\+VM Cost
\item working\+Set\+: Size of the Least Squares Problem in every iteration
\item threads\+: It is the number of parallel threads
\item eta\+: Stop criteria
\end{DoxyItemize}

\subsubsection*{P\+S\+I\+R\+W\+LS algorithm\+:}

It trains a semiparametric S\+VM using a parallel I\+R\+W\+LS procedure. See the library \href{https://robedm.github.io/LIBIRWLS/}{\tt webpage} for a detailed description. \begin{DoxyVerb}    model = LIBIRWLS.PSIRWLStrain(data, labels, gamma=1, C=1, threads=1, size=500, algorithm=0.001, kernel=1)
\end{DoxyVerb}


Parameters\+:
\begin{DoxyItemize}
\item data\+: Training set (numpy 2d array)
\item labels\+: Training set labels (the label of every training data numpy array with values +1 and -\/1)
\item kernel\+: kernel type\+:
\begin{DoxyItemize}
\item 0 for Linear kernel u\textquotesingle{}{\itshape v}
\item {\itshape 1 for radial basis function exp(-\/gamma}$\vert$u-\/v$\vert$$^\wedge$2)
\end{DoxyItemize}
\item gamma\+: gamma in the radial basis kernel function
\item C\+: S\+VM Cost
\item threads\+: It is the number of parallel threads
\item size\+: Size of the classifier
\item algorithm\+: Algorithm for centroids selection
\begin{DoxyItemize}
\item 0 -- Random Selection
\item 1 -- S\+G\+MA (Sparse Greedy Matrix Approximation
\end{DoxyItemize}
\end{DoxyItemize}

\subsubsection*{To classify a new dataset\+:}

\begin{DoxyVerb}    predictions = LIBIRWLS.predict(model, data, labels=None, Threads=1, Soft=0)
\end{DoxyVerb}


Parameters\+:
\begin{DoxyItemize}
\item model\+: A model obtained using P\+I\+R\+W\+LS or P\+S\+I\+R\+W\+LS.
\item data\+: A dataset to classify.
\item labels\+: dataset labels (optional)
\item Threads\+: It is the number of parallel threads
\item Soft\+:
\begin{DoxyItemize}
\item 0 Class prediction (the output is +1 or -\/1)
\item 1 Soft output\+: The output after the hard decision that decides the class (useful to use in ensembles with other algorithms).
\end{DoxyItemize}
\end{DoxyItemize}

\subsubsection*{To save a model in a file\+:}

\begin{DoxyVerb}    LIBIRWLS.save(model, filename)
\end{DoxyVerb}


\subsubsection*{To save a model in a file\+:}

\begin{DoxyVerb}    model = LIBIRWLS.load(filename)\end{DoxyVerb}
 